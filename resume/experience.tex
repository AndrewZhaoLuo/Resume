\cvsection{Experience}
\begin{cventries}
    \vspace{-0.6em}
  \cventry
    {Modular AI, Core Technologies}
    {Senior Compiler Engineer}
    {Seattle, WA}
    {Jun. 2023 - Now}
    {
      \begin{cvitems}
        \item {Top contributor in the past year (by commits) on a team of 10 to a MLIR based compiler for deep learning models.}
		\item {Major contributor to shape propagation, quantization, and programmability systems of compiler.}
		\item {Lead efforts around inter-operability with Mojo with a group of 7 engineers in creating a Triton-lang like experience.}
	\vspace{-1em}
      \end{cvitems}
    }

  \vspace{-0.4em}
  \cventry
    {Octoml, Applied Compiler Engineering - Performance}
    {Staff Engineer}
    {Seattle, WA}
    {Mar. 2021 - Jun. 2023}
    {
      \begin{cvitems}
        \item {\underline{\textbf{Official committer to Apache TVM (9.2k stars on GitHub)}}, an open source autotuning compiler written in C++ and Python.}
		\item {Contribute quantization, mixed precision, ONNX support in TVM. Write kernels and benchmarking tools to improve iteration.}
		\item {Implementing features in SaaS product such as compiler cache, adding new tuning algorithms, and lowering resource usage.}
	\vspace{-1em}
      \end{cvitems}
    }


  \cventry
    {Apple, AI/ML Machine Intelligence Neural Design}
    {Machine Learning Engineer}
    {Seattle, WA}
    {Jan. 2020 - Mar. 2021}
    {
      \begin{cvitems}
        \item {Using quantization, sparsity, and hardware-specific knowledge to train models for Siri, Homepod, and future products}
        \item {Developing in-house solutions for training vision models and deploying/benchmarking on FPGA and ASIC environments}
		\item {Developing demos show-casing technologies to internal stake-holders.}
	\vspace{-1em}
      \end{cvitems}
    }

  \vspace{-0.4em}
  \cventry
    {XNOR.ai, Machine Learning Team (Acquired by Apple)}
    {}
    {}
    {Aug. 2019 - Jan 2020}
    {
      \begin{cvitems}
        \item {Training performant computer vision models that can run on bespoke and edge hardware. Part time until Jan 2019.}
        \item {Creating demos showcasing XNOR's technologies to key executives at major tech companies}. 
	\vspace{-1em}
      \end{cvitems}
    }

\begin{comment}
  OLD EXPERIENCES
  \cventry
    {Sift Science, Core Data}
    {Engineering Intern}
    {San Francisco, CA}
    {Jun. 2018 - Sep. 2018}
    {
      \begin{cvitems}
        \item {Rewrote HBase snapshot system, saving over \$1.5 million in S3 costs a year. Added BigQuery integration with HBase.}
      \end{cvitems}
    }

    \vspace{-0.3em}
  \cventry
    {Mode Lab, University of Washington CSE}
    {Undergraduate Research Assistant}
    {Seattle, WA}
    {Mar. 2018 - Sep. 2018}
    {
      \begin{cvitems}
        \item {Studying the application of linear dynamical systems (LDS) on creating sparse models for MEG data.}
        \item {Implemented stochastic gradient descent for inference of LDS using \textit{Python, numpy, Autograd, einsum2}}
      \end{cvitems}
    }
    \vspace{-1.2em}
  \cventry
    {Facebook, Ads Core}
    {Software Engineering Intern}
    {Seattle, WA}
    {Jun. 2017 - Sep. 2017}
    {
      \begin{cvitems}
    \item {Implemented back-end statistical models to predict demographics of ad reach for customers with multi-million yearly spend}
    %\item {Built data ingestion and ETL pipelines to create training data sets with specific properties}
    %\item {Languages: \textit{Java, Python}. Technologies: \textit{Hive, Presto, Dataswarm}}
      \end{cvitems}
    }
    \vspace{-1.2em}

  \cventry
    {CSE312 (probability for CS) and CSE446 (introduction to ML), University of Washington CSE}
    {Undergraduate Teaching Assistant}
    {Seattle, WA}
    {Sep. 2018 - Jun. 2019}
    {
      \begin{cvitems}
	    \item {Gave weekly lectures to 20-30 students, hold weekly office hours, graded homework, created answer keys and new material}
      \end{cvitems}
    }
    \vspace{-1.2em}

  \cventry
    {Ubiquitous Computing Laboratory, University of Washington CSE}
    {Undergraduate Research Assistant}
    {Seattle, WA}
    {Feb. 2016 - Jan. 2018}
    {
      \begin{cvitems}
        \item {Created ML models for error detection in spirometry, inferring lung health from audio, exposed REST api to use models}
	    \item {Created site in Django for collecting and labeling data, met with doctors monthly to coordinate efforts}
        \item {Languages: \textit{Python}. Technologies: \textit{Scikit-learn, Pandas, Tensorflow, Django}}
      \end{cvitems}
    }
    \vspace{-1.2em}
    
  \cventry
    {Schmulevich group, Institute For Systems Biology}
    {Computational Biology Intern}
    {Seattle, WA}
    {Jun. 2016 - Aug 2016}
    {
      \begin{cvitems}
        \item {Utilized python to create genetic simulation pipeline for evaluating gene set analysis algorithms}
	      \item {TA'd 2-week workshop on utilizing machine learning to characterize cancers using biomarkers}
      \end{cvitems}
    }
    \vspace{-0.7em}


  \cventry
    {Zheng Lab}
    {Undergraduate Research Assistant}
    {Seattle, WA}
    {Sept. 2015 - March 2016}
    {
      \begin{cvitems}
        \item {Modified a 3D printer to extrude and build molten sugar glass through pneumatic heated syringe}
	      \item {Programmed printer to print latices with interesting geometries for vascular biology research}
      \end{cvitems}
    }
    \vspace{-0.7em}
\end{comment}

\end{cventries}
